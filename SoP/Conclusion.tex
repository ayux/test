\section{Conclusion}

An \ac{SoP} multimodal database will be made available to researchers allowing the study of the \ac{IL} of a content as well as the impact of the device changes during a content visualization. The ultimate objective is to provide a reliable database leading to the supply of an higher quality multimedia experiences independently of the \ac{QoS}.

The \ac{SoP} is assessed thanks to the brain and heart activity as well as the respiration (body response such as sympathetic and parasympathetic activity) and subjects subjective ratings (possibly skewed by subjects related factors such as culture, education etc.)

The analysis of the database demonstrates that all the \ac{IL} were experienced with a clear distinction between low and high immersive experiences during multimedia contents visualization.
A shallow processing of the database (19 \ac{EEG} electrode only) was conducted. This latter leads to the conclusion [CONCLUSION CLASSIFIERS].