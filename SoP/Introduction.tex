\section{Introduction}
As digital television technologies aim to provide higher quality multimedia experiences, possibly with various \ac{QoS}, the \ac{SoP} should be investigated to understand its impact on the \acp{QoE}.

According to \cite{SS}, the \ac{SoP} also called immersiveness level refers to the sense of a user that he\/she as left the real world and is now "present" in a virtual environment. A database on the \ac{SoP} aims at studying the impact of immersive experience on the \ac{QoE},in using multimedia contents.

Traditionally, the perceived \ac{SoP} of a multimedia content is estimated based on subjective assessments, where perceived \ac{SoP} of selected visual stimuli is obtained from various subjects. In particular the subjects have to explicitly rate the quality of each stimulus in a pre-defined rating scale. 

The subjective immersiveness perception is, for instance, cultural and educational-dependent. Thus the subjective ratings contain a subjective bias. A valuable survey of the \ac{SoP} can not only rely on subjective ratings. Based on \cite{2Dvs3D}, subjects'physiological signals such as brain activity (\ac{EEG}), heart activity (\ac{ECG}) and respiration are objective data adequate to assess the \ac{SoP} complementarily to the subjective rates.

[JUSTIFICATION DATABASE]

This paper presents a novel database that captures the differences in user-experience during multimedia stimuli with various immersiveness levels. \ac{EEG} and peripheral physiological signals including \ac{ECG} and respiration, as well as subjective ratings are required during the experience.

An investigation of the experience transcribed in the subjective ratings allows to associate some \ac{QoS} properties to the \ac{SoP}. Finally the construction of a subject-independent classification system distinguishes the various immersiveness levels based on \ac{EEG} and/or peripheral physiological signals.

The remainder of this paper is organized as follows.
The next section describes how we conducted experiments to collect subjective ratings and physiological responses. Section 3 presents the results of subjective rating analysis and user-independent physiological classification. Finally, conclusion is given in Section 4.