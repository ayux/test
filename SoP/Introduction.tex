\section{Introduction}
As digital television technologies aim to provide higher quality multimedia experience, possibly with various \ac{QoS}, the \ac{SoP} should be investigated to understand its impact on the \ac{QoE}.

According to Sadowski \& Stanney (2002), the \ac{SoP} also called immersiveness level, whether physical or psychological in nature, allows the sense of belief that the user as left the real world and is now "present" in a virtual environment. The aim to create a database on the \ac{SoP} is to study the impact of an high immersiveness experience on the \ac{QoE}, especially during the visualization of a multimedia content.

Traditionally, the perceived quality of a multimedia content is measured thanks to subjective quality assessment, where perceived quality of selected visual stimuli is obtained from a number of subjects. The subjects have to explicitly rate the quality of each stimulus in a pre-defines rating scale. This procedure is replicated by directing the questionnaire on the immersivennes level experienced during a stimulus.

The expression way of each people is specific and for instance cultural and educational dependent. Thus the subjective ratings contains a subjective bias. A valuable survey of the \ac{SoP} can not only rely on subjects subjective rates. Based on the Kroupi results on the assessment of 2D vs 3D quality [TO SET], subjects'biological signals such as brain activity (\ac{EEG}), heart activity (\ac{ECG}) and respiration are objective data adequate to assess the \ac{SoP} complementarily to the subjective rates.

This paper assesses the creation of a database representing the differences in users'experiences during low, middle and high immersivness level stimuli visualization through \ac{EEG}, and peripheral physiological signals including \ac{ECG} and respiration as well as subjective rates.

We conduct subjective experiments, in which various immersivness stimuli multimedia content were presented to users, and both explicit subjective ratings and implicit \ac{EEG} and physiological response were recorded.
\\Then an investigation of the felt experience transcribed in the subjective ratings allows to associate some \ac{QoS} properties to the \ac{SoP}. Finally the construction of a subject-independent classification system distinguishing the various immersiveness levels of stimuli based on \ac{EEG} and/or peripheral physiological signals.

The remainder of this paper is organized as follows.
The next section describes how we conducted experiments to collect subjective ratings and physiological responses. Section 3 presents the results of subjective rating analysis and user-independent physiological classification. Finally, conclusion is given in Section 4.