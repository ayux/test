\section{Introduction}
%As digital television technologies aim to provide higher quality multimedia experiences, possibly with various \ac{QoS}. In virtual environments, a quality metric is the \ac{SoP} should be investigated to understand its impact on the \acp{QoE}.\cite{Dinh:1999:EIM:554230.835733,6790390}[TBD] %Sanchez-Vives2005-SANFPT,vonderPütten2012317,
%
%According to \cite{SS} the \ac{SoP}, also called immersiveness level in this paper, refers to the subjective experience of having left the real world and being now "present" in a virtual environment. A database on the \ac{SoP} aims at studying the impact of immersive experience on the \ac{QoE},in using multimedia contents.
%
%Traditionally, the perceived \ac{SoP} of a multimedia content is estimated based on subjective assessments, where perceived \ac{SoP} of selected visual stimuli is obtained from various subjects. In particular the subjects have to explicitly rate the quality of each stimulus in a pre-defined rating scale as recommended in \cite{ACRevaluation}. 
%
%The subjective immersiveness perception is, for instance, emotional, cultural and educational-dependent \cite{forgas1999feeling,geng2010cultural,5252120,1979}. Thus the subjective ratings contain a subjective bias. A valuable survey of the \ac{SoP} can not only rely on subjective ratings. Based on \cite{5871728,2Dvs3D,6095505}, subjects'physiological signals such as brain activity (\ac{EEG}), heart activity (\ac{ECG}) and respiration are objective data adequate to assess the \ac{SoP} complementarily to the subjective rates.
%
%[JUSTIFICATION DATABASE + ]
%
%This paper presents a novel database that captures the differences in user-experience during multimedia stimuli with various immersiveness levels. \ac{EEG} and peripheral physiological signals including \ac{ECG} and respiration, as well as subjective ratings are required during the experience.
%
%An investigation of the experience transcribed in the subjective ratings allows to associate some \ac{QoS} properties to the \ac{SoP}. Finally the construction of a subject-independent classification system distinguishes the various immersiveness levels based on \ac{EEG} and/or peripheral physiological signals.
%
%The remainder of this paper is organized as follows.
%The next section describes how we conducted experiments to collect subjective ratings and physiological responses. Section 3 presents the results of subjective rating analysis and user-independent physiological classification. Finally, conclusion is given in Section 4.
%


The \ac{SoP} is the main quality metric for virtual environment as attested in \cite{sw97,Dinh:1999:EIM:554230.835733}.
According to \cite{SS}, the \ac{SoP}, also called immersiveness level in this paper, refers to the subjective experience of having left the real world and being now "present" in a virtual environment.
Its definition is explicated in \cite{Witmer:1998:MPV:1246761.1246762,6790390} which also explains how to measure it.

The human subjective and explicit assessment is, for instance, emotional, cultural and educational-dependent \cite{forgas1999feeling,geng2010cultural,5252120,1979}. Thus a valuable survey of the \ac{SoP} can not only rely on subjective ratings. Based on \cite{5871728,2Dvs3D,6095505}, subjects'physiological signals such as brain activity (\ac{EEG}), heart activity (\ac{ECG}) and respiration are objective data adequate to assess the \ac{SoP} complementarily to the subjective rates.

Multimedia content provided for TV can be viewed as small virtual environment. Indeed, sensory cues for virtual environment usually consists primarily of visual stimuli, often but not always accompanied with audio stimuli. 
Digital television technologies aim to provide higher quality multimedia experiences, possibly with various \ac{QoS}. Therefore the \ac{SoP} can be investigated to understand and to further analyze the \acp{QoE}.

This paper presents a novel database that captures the differences in user-experience during multimedia stimuli with various immersiveness levels. \ac{EEG} and peripheral physiological signals including \ac{ECG} and respiration, as well as subjective ratings are required during the experiment.

An investigation of the experience transcribed in The subjective ratings analysis shows that some \ac{QoS} properties are correlated with the \ac{SoP}. Finally the constructed subject-independent classification system distinguishes the various immersiveness levels based on \ac{EEG} and/or peripheral physiological signals.

The remainder of this paper is organized as follows.
The next section describes how we conducted experiments to collect subjective ratings and physiological responses. Section 3 presents the results of subjective rating analysis and user-independent physiological classification. Finally, conclusion is given in Section 4.


