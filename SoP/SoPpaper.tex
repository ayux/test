% This is "sig-alternate.tex" V2.0 May 2012
% This file should be compiled with V2.5 of "sig-alternate.cls" May 2012
%
% This example file demonstrates the use of the 'sig-alternate.cls'
% V2.5 LaTeX2e document class file. It is for those submitting
% articles to ACM Conference Proceedings WHO DO NOT WISH TO
% STRICTLY ADHERE TO THE SIGS (PUBS-BOARD-ENDORSED) STYLE.
% The 'sig-alternate.cls' file will produce a similar-looking,
% albeit, 'tighter' paper resulting in, invariably, fewer pages.
%
% ----------------------------------------------------------------------------------------------------------------
% This .tex file (and associated .cls V2.5) produces:
%       1) The Permission Statement
%       2) The Conference (location) Info information
%       3) The Copyright Line with ACM data
%       4) NO page numbers
%
% as against the acm_proc_article-sp.cls file which
% DOES NOT produce 1) thru' 3) above.
%
% Using 'sig-alternate.cls' you have control, however, from within
% the source .tex file, over both the CopyrightYear
% (defaulted to 200X) and the ACM Copyright Data
% (defaulted to X-XXXXX-XX-X/XX/XX).
% e.g.
% \CopyrightYear{2007} will cause 2007 to appear in the copyright line.
% \crdata{0-12345-67-8/90/12} will cause 0-12345-67-8/90/12 to appear in the copyright line.
%
% ---------------------------------------------------------------------------------------------------------------
% This .tex source is an example which *does* use
% the .bib file (from which the .bbl file % is produced).
% REMEMBER HOWEVER: After having produced the .bbl file,
% and prior to final submission, you *NEED* to 'insert'
% your .bbl file into your source .tex file so as to provide
% ONE 'self-contained' source file.
%
% ================= IF YOU HAVE QUESTIONS =======================
% Questions regarding the SIGS styles, SIGS policies and
% procedures, Conferences etc. should be sent to
% Adrienne Griscti (griscti@acm.org)
%
% Technical questions _only_ to
% Gerald Murray (murray@hq.acm.org)
% ===============================================================
%
% For tracking purposes - this is V2.0 - May 2012

\documentclass{sig-alternate}
\usepackage{acronym}

\begin{document}

% acronym definitions
\newacro{SoP}{Sense of Presence}
\newacro{QoE}{Quality of experience}
\newacro{QoS}{Quality of service}
\newacro{EEG}{electroencephalography}
\newacro{ECG}{electrocardiography}
\newacro{Resp.}{respiration}

%
% --- Author Metadata here ---
\conferenceinfo{ACM MM}{2015 , Brisbane, Australia}
%\CopyrightYear{2007} % Allows default copyright year (20XX) to be over-ridden - IF NEED BE.
%\crdata{0-12345-67-8/90/01}  % Allows default copyright data (0-89791-88-6/97/05) to be over-ridden - IF NEED BE.
% --- End of Author Metadata ---

\title{Sense of Presence Multimodal Dataset}
%\subtitle{[Extended Abstract]
%\titlenote{A full version of this paper is available as
%\textit{Author's Guide to Preparing ACM SIG Proceedings Using
%\LaTeX$2_\epsilon$\ and BibTeX} at
%\texttt{www.acm.org/eaddress.htm}}}

% For aesthetic reasons, we recommend 'three authors at a time'
% i.e. three 'name/affiliation blocks' be placed beneath the title.



\numberofauthors{1} 
\author{
%\alignauthor 
Anne-Flore Perrin, He Xu, Eleni Kroupi, Martin Rerabek, Touradj Ebrahimi \\
       \affaddr{Multimedia Signal Processing Group (MMSPG)}\\
       \affaddr{Ecole Polytechnique Federale de Lausanne (EPFL)}\\
       \affaddr{EPFL/STI/IEL/GR-EB, Station 11, CH-1015 Lausanne, Switzerland}\\
       \email{\{anne-flore.perrin, he.xu, eleni.kroupi, martin.rerabek, touradj.ebrahimi\}@epfl.ch}
%% 1st. author
%\alignauthor Anne-Flore Perrin \\
%       \affaddr{Multimedia Signal Processing Group (MMSPG)}\\
%       \affaddr{Ecole Polytechnique Federale de Lausanne (EPFL)}\\
%       \affaddr{EPFL/STI/IEL/GR-EB, Station 11, CH-1015 Lausanne, Switzerland}\\
%       \email{anne-flore.perrin@epfl.ch}
%%% 2nd. author
%\alignauthor He Xu\\
%       \affaddr{Multimedia Signal Processing Group (MMSPG)}\\
%       \affaddr{Ecole Polytechnique Federale de Lausanne (EPFL)}\\
%       \affaddr{EPFL/STI/IEL/GR-EB, Station 11, CH-1015 Lausanne, Switzerland}\\
%       \email{he.xu@epfl.ch}
%%% 3rd. author
%\alignauthor Eleni Kroupi\\
%       \affaddr{Multimedia Signal Processing Group (MMSPG)}\\
%       \affaddr{Ecole Polytechnique Federale de Lausanne (EPFL)}\\
%       \affaddr{EPFL/STI/IEL/GR-EB, Station 11, CH-1015 Lausanne, Switzerland}\\
%       \email{eleni.kroupi@epfl.ch}
%%% 4th. author
%\alignauthor Martin Rerabek\\
%       \affaddr{Multimedia Signal Processing Group (MMSPG)}\\
%       \affaddr{Ecole Polytechnique Federale de Lausanne (EPFL)}\\
%       \affaddr{EPFL/STI/IEL/GR-EB, Station 11, CH-1015 Lausanne, Switzerland}\\
%       \email{martin.rerabek@epfl.ch}
%%% 5th. author
%\alignauthor Touradj Ebrahimi\\
%       \affaddr{Multimedia Signal Processing Group (MMSPG)}\\
%       \affaddr{Ecole Polytechnique Federale de Lausanne (EPFL)}\\
%       \affaddr{EPFL/STI/IEL/GR-EB, Station 11, CH-1015 Lausanne, Switzerland}\\
%       \email{touradj.ebrahimi@epfl.ch}
%% 6th. author
%\alignauthor Charles Palmer\\
%       \affaddr{Palmer Research Laboratories}\\
%       \affaddr{8600 Datapoint Drive}\\
%       \affaddr{San Antonio, Texas 78229}\\
%       \email{cpalmer@prl.com}
}
% There's nothing stopping you putting the seventh, eighth, etc.
% author on the opening page (as the 'third row') but we ask,
% for aesthetic reasons that you place these 'additional authors'
% in the \additional authors block, viz.
%\additionalauthors{Additional authors: John Smith (The Th{\o}rv{\"a}ld Group,
%email: {\texttt{jsmith@affiliation.org}}) and Julius P.~Kumquat
%(The Kumquat Consortium, email: {\texttt{jpkumquat@consortium.net}}).}
\date{30 April 2015}
% Just remember to make sure that the TOTAL number of authors
% is the number that will appear on the first page PLUS the
% number that will appear in the \additionalauthors section.

\maketitle
\begin{abstract}
Measuring the quality of a set of videos with various properties, as the content, quality, resolution and sound system, is not an easy task because of the subjectivity of the human perception. A way to express the quality of a video is to assess the \ac{QoE} and the \ac{SoP}. 
The second one, not so explored, is the principle we would like to assess. To be independent of the human way to express the feelings and experiences, the measure is built thanks to biological signals ( EEG, ECG and respiration).
\\The project is to create a data set containing all the biological signals enabling the SoP study, process those data and demonstrate the data set is well constructed and functional for the assessment of the sense of presence for instance.

\end{abstract}

%% A category with the (minimum) three required fields
%\category{H.4}{Information Systems Applications}{Miscellaneous}
%%A category including the fourth, optional field follows...
%\category{D.2.8}{Software Engineering}{Metrics}[complexity measures, performance measures]
%
%\terms{Theory}
%
%\keywords{ACM proceedings, \LaTeX, text tagging}

\acresetall
\section{Introduction}
%As digital television technologies aim to provide higher quality multimedia experiences, possibly with various \ac{QoS}. In virtual environments, a quality metric is the \ac{SoP} should be investigated to understand its impact on the \acp{QoE}.\cite{Dinh:1999:EIM:554230.835733,6790390}[TBD] %Sanchez-Vives2005-SANFPT,vonderPütten2012317,
%
%According to \cite{SS} the \ac{SoP}, also called immersiveness level in this paper, refers to the subjective experience of having left the real world and being now "present" in a virtual environment. A database on the \ac{SoP} aims at studying the impact of immersive experience on the \ac{QoE},in using multimedia contents.
%
%Traditionally, the perceived \ac{SoP} of a multimedia content is estimated based on subjective assessments, where perceived \ac{SoP} of selected visual stimuli is obtained from various subjects. In particular the subjects have to explicitly rate the quality of each stimulus in a pre-defined rating scale as recommended in \cite{ACRevaluation}. 
%
%The subjective immersiveness perception is, for instance, emotional, cultural and educational-dependent \cite{forgas1999feeling,geng2010cultural,5252120,1979}. Thus the subjective ratings contain a subjective bias. A valuable survey of the \ac{SoP} can not only rely on subjective ratings. Based on \cite{5871728,2Dvs3D,6095505}, subjects'physiological signals such as brain activity (\ac{EEG}), heart activity (\ac{ECG}) and respiration are objective data adequate to assess the \ac{SoP} complementarily to the subjective rates.
%
%[JUSTIFICATION DATABASE + ]
%
%This paper presents a novel database that captures the differences in user-experience during multimedia stimuli with various immersiveness levels. \ac{EEG} and peripheral physiological signals including \ac{ECG} and respiration, as well as subjective ratings are required during the experience.
%
%An investigation of the experience transcribed in the subjective ratings allows to associate some \ac{QoS} properties to the \ac{SoP}. Finally the construction of a subject-independent classification system distinguishes the various immersiveness levels based on \ac{EEG} and/or peripheral physiological signals.
%
%The remainder of this paper is organized as follows.
%The next section describes how we conducted experiments to collect subjective ratings and physiological responses. Section 3 presents the results of subjective rating analysis and user-independent physiological classification. Finally, conclusion is given in Section 4.
%


The \ac{SoP} is the main quality metric for virtual environment as attested in \cite{sw97,Dinh:1999:EIM:554230.835733}.
According to \cite{SS}, the \ac{SoP}, also called immersiveness level in this paper, refers to the subjective experience of having left the real world and being now "present" in a virtual environment.
Its definition is explicated in \cite{Witmer:1998:MPV:1246761.1246762,6790390} which also explains how to measure it.

The human subjective and explicit assessment is, for instance, emotional, cultural and educational-dependent \cite{forgas1999feeling,geng2010cultural,5252120,1979}. Thus a valuable survey of the \ac{SoP} can not only rely on subjective ratings. Based on \cite{5871728,2Dvs3D,6095505}, subjects'physiological signals such as brain activity (\ac{EEG}), heart activity (\ac{ECG}) and respiration are objective data adequate to assess the \ac{SoP} complementarily to the subjective rates.

Multimedia content provided for TV can be viewed as small virtual environment. Indeed, sensory cues for virtual environment usually consists primarily of visual stimuli, often but not always accompanied with audio stimuli. 
Digital television technologies aim to provide higher quality multimedia experiences, possibly with various \ac{QoS}. Therefore the \ac{SoP} can be investigated to understand and to further analyze the \acp{QoE}.

This paper presents a novel database that captures the differences in user-experience during multimedia stimuli with various immersiveness levels. \ac{EEG} and peripheral physiological signals including \ac{ECG} and respiration, as well as subjective ratings are required during the experiment.

An investigation of the experience transcribed in The subjective ratings analysis shows that some \ac{QoS} properties are correlated with the \ac{SoP}. Finally the constructed subject-independent classification system distinguishes the various immersiveness levels based on \ac{EEG} and/or peripheral physiological signals.

The remainder of this paper is organized as follows.
The next section describes how we conducted experiments to collect subjective ratings and physiological responses. Section 3 presents the results of subjective rating analysis and user-independent physiological classification. Finally, conclusion is given in Section 4.


 
\section{Data Collection}

\subsection{Participants}
The height females and the twelve males participants had from 18 to 30 years old (23 median and average years of age). The 20 subject were screened for correct visual acuity (no errors on 20/30 lines) and color vision using Snellen and Ishiara charts respectively. They all provided written consents forms. Before each experiment, oral instruction were provided to the participants to explain their tasks. Additionally, a training session was organized to allow participants to familiarize with the assessment procedure. The content shown in the training session was selected by experts viewers in order to include examples of all evaluated aspects.

\subsection{Audio-visual stimuli}
Video stimuli are coming from nine video sequences coming from four open source movies published by the Blender Foundation (Big buck bunny, Elephant dream, Sintel and Tears of Steel)\footnote{$http://media.xiph.org/$}. A supplementary sequence content was chosen for the training session.
\\The one minutes selected video contents have the highest audio, spatial end temporal energy and are related to the scene cuts.
The twenty seven video stimuli shown during an experiment are the combination of the nine video content with the three levels of immersiveness described below.
\\Low, middle and high immersiveness level were defined respectively thanks to the audio sound system, the video quality\textbackslash level of compression and the resolution.
The table \ref{IL} illustrates the previous comment.

\begin{table}[h]
\begin{tabular}{ |c || c | c | c | }
   \hline	
   Immersive senario 	& Low 			& Middle 		& High \\
   \hline	
   Audio 				& No Audio 		& Stereo		& Surround \\
   Quality (QP) 		& 36 			& 20			& 20 \\
   Resolution			& SD			& HD			& UHD\\
   \hline	
 \end{tabular}
 \caption{immersiveness levels}
 \label{IL}
 \end{table}

The video stimuli order could impact the data and so the results. Thus
the sessions are built to allow this study : The first session will display
the video stimuli from the the lowest immersiveness level stimuli to ones
with the highest. The second session order is middle immersiveness level
stimuli, followed by the low immersiveness level stimuli and then the high
immersiveness level stimuli. The last session will display the video stimuli
from the video with the highest immersiveness level to ones with the lowest.
Besides these constraints, the video order is different for each volunteer,
thanks to a pseudo-random function.

\subsection{Monitor, sound system and environment}

Professional high-performance 4K/QFHD LCD reference 56-inch monitor Sony Trimaster SRM-L560\footnote{$http://pro.sony.com/bbsccms/assets/files/cat/mondisp/$ \newline $brochures/di0195\_srm1560.pdf$} was used to display video stimuli.
As recommended in [TO SET comparsing upscaling algorithms from HD to UHD...] the viewing distance was set at 1.6H (H - Height of the screen).
[SOUND SYSTEM]


\subsection{Physiological signal acquisition}
The nets used for the recording of the \ac{EEG} signals contains 256 electrodes placed at the standard position on the scalp. An EGI's Geodesic EEG System (GES) 300 was used to record, amplify, and digitized the EEG signals while the participants were watching the stimuli. Additionally, two standard ECG leads were placed on the lower left ribcage and on the upper right clavicle, as well as two respiratory inductive plethysmography belts (thoracic and abdomen). All signals were recorded at 250 Hz.


\subsection{Experimental protocol}

The experiments were conducted in three sessions. A ten-minutes break was given between two sessions in order to avoid subject fatigue and lack of attention. Nine video sequences were presented in each session leading to a total of 27 video sequences, and thus, to a total of 27 trials.
Each trial consisted of a ten-second baseline period and a stimulus period. The biosignals recorded during the baseline period were used to remove stimulus-unrelated variations from the signals obtained during the stimulus period.

During the baseline periods, the subjects were instructed to remain calm and focus on a 2D white cross on a black background presented on the screen in front of them. Once this baseline period was over, a video sequence was pseudo-randomly selected and presented.
After the video sequence was over, the subjects were asked to provide their self-assessed ratings for the particular video sequence without any restriction in time, following the Absolute Category Rating (ACR) evaluation methodology [TO SET]

Regarding the self-assessed ratings, subjects were asked to evaluate the video sequences in terms of five different aspects, namely interest in the video content, perceived video quality, interest in audio content, immersiveness level and surrounding awareness. A 9-point rating scale was used that ranged from 1 to 9, with 1 representing the lowest value, and 9 the hightst value of each aspect. In particular, the two extremes (1 and 9) correspond to "low" and "high" for interest in video and audio content as well as the perceived video quality, "no immersion" and "full immersion" for the immersion level and "full conscience of my environment" and "no conscience of my environment" for the surrounding awareness.

Once a trial was over, the next baseline period was recorded and the next video sequence was pseudo-randomly selected and presented. The procedure was repeated until all 27 video stimuli were presented and rated, leading to 27 trials. Although the experiments lasted for almost two hours, including the training and the set up, the subjects did not report fatigue.

An illustration of a session is presented figure \ref{session}.

\begin{figure}[!ht]
    \center
    \includegraphics[width=0.5\textwidth]{./images/ExSession.png}
    \caption{Example of a 3 video stimuli session progress }
    \label{session}
\end{figure}









 
%\input{ExperimentalProtocol} 
\section{Analysis}

\subsection{Subjective ratings analysis}

\subsection{Physiological signal analysis}

\subsubsection{Pre-processing}

\subsubsection{Correlation} 
\section{Conclusion}

An \ac{SoP} multimodal database will be made available to researchers allowing the study of the \ac{IL} of a content as well as the impact of the device changes during a content visualization. The ultimate objective is to provide a reliable database leading to the supply of an higher quality multimedia experiences independently of the \ac{QoS}.

The \ac{SoP} is assessed thanks to the brain and heart activity as well as the respiration (body response such as sympathetic and parasympathetic activity) and subjects subjective ratings (possibly skewed by subjects related factors such as culture, education etc.)

The analysis of the database demonstrates that all the \ac{IL} were experienced with a clear distinction between low and high immersive experiences during multimedia contents visualization.
A shallow processing of the database (19 \ac{EEG} electrode only) was conducted. This latter leads to the conclusion [CONCLUSION CLASSIFIERS].

\end{document}
